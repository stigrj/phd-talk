\begin{frame}
    \centering
    \Large{Part I:}\\
    \ \\
    \ \\
    \centering
    \Large{Adaptive order polynomial algorithm in a\\
	    multiwavelet representation scheme}
\end{frame}

\begin{frame}
    \frametitle{Multiwavelets}
    \begin{columns}
    \begin{column}[b]{0.55\linewidth}
	\begin{itemize}
	    \item   \textbf{Scaling functions} are polynomials or order $\leq k$\\
		    on the unit interval
	    \item   \textbf{Dilation and translation} to refinement scale $n$
		    \begin{equation}
			\nonumber
			\phi_l^n(x) = 2^{n/2}\phi(2^nx-l)
		    \end{equation}
	    \item   At scale $n$ there are $2^n$ subintervals
	    \item   The length of each subinterval is $2^{-n}$
	    \item   \textbf{Scaling projection} at scale $N$
		    \begin{equation}
			\nonumber
			f(x) \approx f^N(x) = \sum_l s_l^N \phi_l^N(x)
		    \end{equation}
		    \ \\
		    \ \\
	    \item   For a $k$-order basis in $d$ dimensions there are\\
		    $\left(2(k+1)\right)^{nd}$ basis functions at scale $n$\\
	\end{itemize}
	\centering
	\includegraphics[scale=0.2]{figures/unifgrid.pdf}
    \end{column}
    \begin{column}[b]{0.45\linewidth}
	\centering
	\includegraphics[scale=0.3, clip, viewport = 150 450 450 750]{figures/scaling.pdf}\\
	\includegraphics[scale=0.3, clip, viewport = 100 400 500 800]{figures/refinement.pdf}
    \end{column}
    \end{columns}
\end{frame}

\begin{frame}
    \frametitle{Multiwavelets}
    \centering
    \only<1>{\ \ \ \includegraphics[clip, viewport=50 100 600 800, scale=0.5]{figures/f0.pdf}}
    \only<2>{\ \ \includegraphics[clip, viewport=50 100 600 800, scale=0.5]{figures/f1.pdf}}
    \only<3>{\ \includegraphics[clip, viewport=50 100 600 800, scale=0.5]{figures/f2.pdf}}
    \only<4>{\includegraphics[clip, viewport=50 100 600 800, scale=0.5]{figures/f3.pdf}}
    \only<5>{\includegraphics[clip, viewport=50 100 600 800, scale=0.5]{figures/f4.pdf}}
\end{frame}

\begin{frame}
    \frametitle{Multiwavelets}
    \begin{columns}
    \begin{column}[b]{0.55\linewidth}
	\begin{itemize}
	    \item   \textbf{Wavelet functions} are piecewise polynomials
	    \item   \textbf{Wavelet projection} at scale $N$
		    \begin{equation}
			\nonumber
			df^n(x) = f^{n+1}(x) - f^{n}(x)
		    \end{equation}
		    \ \\
	    \item   Alternative \textbf{multiresolution} representation
		    \begin{equation}
			\nonumber
			f^N(x) = f^{0}(x) + \sum_{n=0}^{N-1} df^{n}(x)
		    \end{equation}
	    \item   Allows for \textbf{adaptive refinement} by local thresholding
		    \begin{equation}
			\nonumber
			\|df_l^n\| < \frac{\epsilon}{2^{n/2}}\|f\|
		    \end{equation}
	    \item   Representations with \textbf{guaranteed precision} $\epsilon$
	\end{itemize}
	\ \\
	\ \\
    \end{column}
    \begin{column}[b]{0.45\linewidth}
	\centering
	\includegraphics[scale=0.3, clip, viewport = 100 400 500 800]{figures/adaptivity.pdf}
    \end{column}
    \end{columns}
    \ \\
    \centering
    \includegraphics[scale=0.2]{figures/unifgrid.pdf}
    \includegraphics[scale=0.1192]{figures/adapgrid.pdf}
\end{frame}


\begin{frame}
    \frametitle{Decreasing order polynomial basis}
    \begin{columns}
    \begin{column}[b]{0.7\linewidth}
	\begin{itemize}
	    \item   Multi-dimensional functions require many expansion coefficients\\
		    \ \\
	    \item   Try to reduce the memory footprint by varying the polynomial order\\
		    \ \\
	    \item   Specifically decreasing the order $k$ with increasing scale $n$
	\end{itemize}
	\ \\
	\ \\
	\ \\
    \end{column}
    \begin{column}[b]{0.3\linewidth}
    \centering
    \begin{figure}
	\setlength{\unitlength}{.5mm}
	\begin{picture}(93,46)
	    \put( 0,10){\vector(1,0){60}} %n- axis
	    \put(61,10){$n$}
	    \put(10,4){\vector(0,1){37}} %k-axis
	    \put(10,43){$k$}
	    \put(40,38){$k(n)$}
	    \put(-5,30){$k_{\max}$}
	    \put(-5,15){$k_{\min}$}
	    \put(25,5){$n_0$}
	    \put(40,5){$n_1$}
	    \put(10,30){\line(1,0){15}}
	    \put(25,30){\line(1,-1){15}}
	    \put(40,15){\line(1,0){15}}
	    \multiput(40,10)(0,0.1){5}{\line(0,1){2}} 
	    \multiput(25,10)(0,4){5}{\line(0,1){2}}
	    \multiput(10,15)(6,0){5}{\line(1,0){2}}
	\end{picture}
    \end{figure}
    \end{column}
    \end{columns}
    \ \\
    \centering
    \includegraphics[scale=0.6]{figures/decrease.pdf}
\end{frame}


