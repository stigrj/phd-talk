\begin{frame}
\frametitle{Integral formulation linear response}
\centering
Adding a first order perturbation $\delta\hat{h}$ to the Hamiltonian \\
gives the perturbation in the Fock operator
\begin{equation}
    \nonumber
    \delta\hat{F} = \delta\hat{h} + \delta \hat{V}
\end{equation}

%\vspace{6mm}
%where the changes in the potential operator can have the form
%\begin{align}
%    \nonumber
%    \delta v_{coul}(r) &= \int P(r-r')[\delta\rho(r',r')]dr'\\
%    \nonumber
%    \delta v_{xc}(r) &= 
%        \delta\rho(r,r)\frac{\partial^2 E_{xc}}{\partial\rho^2}(r)\\
%    \nonumber
%    \delta \hat{K} \phi_p(r) 
%    &= \sum_i \phi_i(r) \int P(r-r')[\delta\phi_i^\dag(r')\phi_p(r')]dr' \\
%    \nonumber
%    &+ \sum_i \delta\phi_i(r) \int P(r-r')[\phi_i^\dag(r')\phi_p(r')]dr'
%\end{align}

\vspace{6mm}
where $\delta\hat{V}$ denotes the changes in the potential \\
operator induced by the perturbed density
\begin{equation}
    \nonumber
    \delta \rho(r,r') =
    \sum_i \phi_i(r)\delta\phi_i^\dag(r')\ +\ 
    \delta\phi_i(r)\phi_i^\dag(r')
\end{equation}

\pause

\vspace{6mm}
The orbital perturbations are obtained by solving the modified \\
Sternheimer (coupled perturbed HF/DFT) equations
\begin{equation}
    \nonumber
    \hat{F}\delta\phi_i + \delta\hat{F}\phi_i = 
    \sum_j \Big(F_{ij}\delta\phi_j + \delta F_{ij}\phi_j\Big)
\end{equation}

\vspace{6mm}
which again are inverted to an integral equation and solved iteratively
\begin{equation}
    \nonumber
    \delta\phi_i^{n+1} =\ 
    -2\hat{H}\Big[\hat{V} \delta\phi_i^n
    - \sum_j \big(F_{ij} - \Lambda_{ij}\big)\delta\phi_j^n
    + \Big(1 - \sum_j|\phi_j\rangle\langle\phi_j|\Big) \delta\hat{F}^n\phi_i
    \Big]
\end{equation}

\vspace{5mm}
\tiny
T. Yanai, R.J. Harrison, N.C. Handy,
{\it Mol. Phys.},
\textbf{103:2-3} 
(2005)\\
H. Sekino, Y. Maeda, T. Yanai, R.J. Harrison,
{\it J. Chem. Phys.},
\textbf{129} 
(2008)

\end{frame}


\begin{frame}
\frametitle{Polarizability}
\begin{columns}

\begin{column}[b]{0.5\textwidth}
\centering
\textbf{Electric dipole operator}
\begin{equation}
    \nonumber
    \delta \hat{h} = \boldsymbol{\mu}
\end{equation}
\end{column}


\begin{column}[b]{0.5\textwidth}
\centering
\textbf{Polarizability}
\begin{equation}
    \nonumber
    \alpha 
    %= \sum_i \int \phi_i \Big(\boldsymbol{\mu} \delta\phi_i\Big)^\dag dr 
    %+ \int \Big(\boldsymbol{\mu} \delta \phi_i\Big) \phi_i^\dag dr
    = \int \boldsymbol{\mu} \ \delta\rho \ dr
\end{equation}
\end{column}
\end{columns}

\vspace{5mm}

\centering
Where the perturbed orbitals $\delta\phi_i$ are obtained using 
the $\boldsymbol{\mu}$ operator.

\vspace{5mm}

\begin{table}
%\tiny
\begin{tabular}{l|c|ccc|c}
\multicolumn{6}{c}{\textbf{Hartree-Fock polarizability of $H_2O$ (a.u.)}}\\
\hline
\hline
                    &          &        &        &        &        \\
                    &$E_{tot}$  &$\alpha_{xx}$  &$\alpha_{yy}$  
                    &$\alpha_{zz}$  &$\alpha_{iso}$ \\
                    &          &        &        &        &        \\
%$\epsilon=10^{-3}$  &-76.01459 & 9.4778 & 7.9234 & 8.7537 & 8.7180 \\
MW $\epsilon=10^{-4}$  &-76.06548 & 9.4893 & 7.9698 & 8.7665 & 8.7419 \\
MW $\epsilon=10^{-5}$  &-76.06552 & 9.4887 & 7.9695 & 8.7658 & 8.7413 \\
                    &          &        &        &        &        \\
aug-cc-pV5Z         &-76.06543 & 9.4852 & 7.9503 & 8.7503 & 8.7286 \\
%aug-cc-pVQZ         &-76.06412 & 9.4842 & 7.9169 & 8.7240 & 8.7084 \\
%aug-cc-pVTZ         &-76.05882 & 9.4632 & 7.7941 & 8.6291 & 8.6288 \\
aug-cc-pVDZ         &-76.03980 & 9.3304 & 7.4016 & 8.3028 & 8.3450 \\
                    &          &        &        &        &        \\
%cc-pV6Z             &-76.06551 & 9.2242 & 7.4127 & 8.4060 & 8.3476 \\
cc-pV5Z             &-76.06520 & 9.0604 & 6.8999 & 8.1594 & 8.0399 \\
%cc-pVQZ             &-76.06295 & 8.7732 & 6.1540 & 7.7173 & 7.5482 \\
%cc-pVTZ             &-76.05536 & 8.2921 & 5.1219 & 6.9736 & 6.7959 \\
cc-pVDZ             &-76.02544 & 7.2047 & 3.0348 & 5.3360 & 5.1919 \\
                    &          &        &        &        &        \\
\hline
\hline
\end{tabular}
\end{table}

\it{GTO calculations using LSDalton}

\end{frame}


\begin{frame}
\frametitle{Magnetizability}
\begin{columns}

\begin{column}[b]{0.5\textwidth}
\centering
\textbf{Diamagnetic magnetizability}
\begin{equation}
    \nonumber
    \hat{h}^{dia} = -\frac{1}{4}\Big(r_O^2\boldsymbol{I} - 
    \boldsymbol{r}_O\boldsymbol{r}_O^T\Big)
\end{equation}

\vspace{2mm}

\begin{equation}
    \nonumber
    \chi^{dia} = \sum_i \int \phi_i \hat{h}^{dia} \phi_i dr
\end{equation}
\end{column}

\begin{column}[b]{0.5\textwidth}
\centering
\textbf{Paramagnetic magnetizability}
\begin{equation}
    \nonumber
    \hat{h}^{orb} = -i \boldsymbol{r}_O\times\nabla
\end{equation}
\vspace{2mm}
\begin{equation}
    \nonumber
    \chi^{para} = \sum_i \int \phi_i \Big(\hat{h}^{orb} \delta\phi_i\Big)^\dag + 
    \Big(\hat{h}^{orb}\delta\phi_i\Big)\phi_i^\dag dr
\end{equation}
\end{column}

\end{columns}
\vspace{5mm}
\centering
Where the perturbed orbitals $\delta\phi_i$ are obtained using 
the $\hat{h}^{orb}$ operator.
\vspace{5mm}

\begin{table}
%\tiny
\centering
\begin{tabular}{l|c|ccc}
\multicolumn{5}{c}{\textbf{Hartree-Fock magnetizability of $H_2O$ (a.u.)}}\\
\hline
\hline
                    &         &             &             &            \\
                    &$E_{tot}$&$\chi^{dia}$ &$\chi^{para}$&$\chi^{tot}$\\
                    &         &             &             &            \\
%$\epsilon=10^{-3}$  &-76.01459& -3.2696     & 0.3243      & -2.9453    \\
MW $\epsilon=10^{-4}$  &-76.06548& -3.2688     & 0.3221      & -2.9467    \\
MW $\epsilon=10^{-5}$  &-76.06552& -3.2688     & 0.3220      & -2.9468    \\
                    &         &             &             &            \\
aug-cc-pV5Z         &-76.06543& -3.2690     & 0.3220      & -2.9470    \\
%aug-cc-pVQZ         &-76.06412& -3.2898     & 0.3220      & -2.9479    \\
%aug-cc-pVTZ         &-76.05882& -3.2720     & 0.3220      & -2.9500    \\
aug-cc-pVDZ         &-76.03980& -3.2821     & 0.3248      & -2.9573    \\
       	            &         &             &             &            \\
cc-pV5Z	            &-76.06520& -3.2584     & 0.3234      & -2.9351    \\
%cc-pVQZ	     &-76.06295& -3.2350     & 0.3264      & -2.9087    \\
%cc-pVTZ	     &-76.05536& -3.2047     & 0.3325      & -2.8722    \\
cc-pVDZ	            &-76.02544& -3.1471     & 0.3568      & -2.7902    \\
       	            &         &             &             &            \\
\hline
\hline
\end{tabular}
\end{table}

\it{GTO (GIAOs) calculations using Dalton}

\end{frame}


\begin{frame}
\frametitle{Nuclear shielding}
\begin{columns}

\begin{column}[b]{0.5\textwidth}
\centering
\textbf{Diamagnetic shielding}
\begin{equation}
    \nonumber
    \hat{h}_K^{dia} = \frac{\alpha}{2}\frac{\Big(r_O^Tr_K\boldsymbol{I} - 
    \boldsymbol{r}_O\boldsymbol{r}_O^T\Big)}{r_K^3}
\end{equation}
\vspace{2mm}
\begin{equation}
    \nonumber
    \sigma^{dia} = \sum_i \int \phi_i \hat{h}_K^{dia} \phi_i dr
\end{equation}
\end{column}

\begin{column}[b]{0.5\textwidth}
\centering
\textbf{Paramagnetic shielding}
\begin{equation}
    \nonumber
    \hat{h}_K^{pso} = -i \alpha^2 \frac{\boldsymbol{r}_K\times\nabla}{r_K^3}
\end{equation}
\vspace{2mm}
\begin{equation}
    \nonumber
    \sigma^{para} = \sum_i \int \phi_i \Big(\hat{h}_K^{pso} \delta\phi_i\Big)^\dag + 
    \Big(\hat{h}_K^{pso}\delta\phi_i\Big)\phi_i^\dag dr
\end{equation}
\end{column}

\end{columns}
\vspace{5mm}
\centering
Where the perturbed orbitals $\delta\phi_i$ are obtained using 
the $\hat{h}^{orb}$ operator.
\vspace{5mm}

\begin{table}
%\tiny
\centering
\begin{tabular}{l|lll|ll}
\multicolumn{6}{c}{\textbf{Hartree-Fock shielding tensor of $H_2O$ (ppm)}}\\
\hline
\hline
                    &         &         &         &         &         \\
&\multicolumn{3}{|c}{Hydrogen }&\multicolumn{2}{|c}{Oxygen }\\
                    &         &         &         &         &         \\
                    &$\sigma_{xx}$ &$\sigma_{xz}$&$\sigma_{iso}$ 
                    &$\sigma_{xx}$ &$\sigma_{iso}$\\
                    &         &         &         &         &         \\
%$\epsilon=10^{-3}$  & 37.5045 & 9.4180  & 29.9332 & 357.5210& 318.9106\\
MW $\epsilon=10^{-4}$  & 37.5326 & 9.4862  & 29.9130 & 357.2956& 319.0418\\
MW $\epsilon=10^{-5}$  & 37.5532 & 9.5008  & 29.9220 & 357.1585& 319.0446\\
                    &         &         &         &         &         \\
aug-cc-pV5Z         & 37.5552 & 9.4710  & 29.9477 & 357.0290& 318.9910\\
%aug-cc-pVQZ         & 37.5831 & 9.4309  & 30.0028 & 357.4613& 319.7300\\
%aug-cc-pVTZ         & 37.6995 & 9.3250  & 30.1715 & 357.3193& 320.0901\\
aug-cc-pVDZ         & 37.9432 & 8.9067  & 30.6418 & 362.5730& 328.5298\\
       	            &         &         &         &         &         \\
cc-pV5Z	            & 37.5538 & 9.4822  & 29.9489 & 356.3598& 319.7752\\
%cc-pVQZ	     & 37.6077 & 9.5013  & 30.0495 & 356.5344& 322.6689\\
%cc-pVTZ	     & 37.7956 & 9.4816  & 30.2585 & 354.9871& 326.9582\\
cc-pVDZ	            & 38.1757 & 9.1199  & 30.8304 & 359.0978& 338.6980\\
       	            &         &         &         &         &         \\
\hline
\hline
\end{tabular}
\end{table}

\it{GTO (GIAOs) calculations using Dalton}

\end{frame}


%\begin{frame}
%\frametitle{Origin dependence}
%\centering

%\begin{table}
%\tiny
%\centering
%\begin{tabular}{l|ll|lll}
%\multicolumn{6}{c}{\textbf{Hartree-Fock magnetizability of Helium (a.u.)}}\\
%\hline
%\hline
%                    &         &       &       &      &       \\
%\multicolumn{1}{l}{Gauge origin}&
%\multicolumn{2}{|c}{(0, 0, 0) }&
%\multicolumn{3}{|c}{(0, 0, 5) }\\
%                    &         &       &       &      &       \\
%&
%\multicolumn{1}{c}{$E^{tot}$}&
%\multicolumn{1}{c|}{$\chi^{tot}$}&
%\multicolumn{1}{c}{$\chi^{dia}$}&
%\multicolumn{1}{c}{$\chi^{para}$}&
%\multicolumn{1}{c}{$\chi^{tot}$}\\
%                    &         &       &       &      &       \\
%%$\epsilon=10^{-3}$  &-2.861543&-0.3951&-8.7284&8.3270&-0.4014\\
%$\epsilon=10^{-4}$  &-2.861662&-0.3949&-8.7283&8.3334&-0.3951\\
%$\epsilon=10^{-5}$  &-2.861676&-0.3950&-8.7283&8.3333&-0.3950\\
%                    &         &       &       &      &       \\
%aug-cc-pV6Z         &-2.861673&-0.3951&-8.7283&8.3292&-0.3991\\
%%aug-cc-pV5Z         &-2.861627&-0.3950&-8.7284&8.3222&-0.4062\\
%aug-cc-pVQZ         &-2.861522&-0.3953&-8.7286&8.3157&-0.4129\\
%%aug-cc-pVTZ         &-2.861183&-0.3957&-8.7290&8.2806&-0.4484\\
%aug-cc-pVDZ	    &-2.855705&-0.3990&-8.7324&8.0633&-0.6691\\
%                    &         &       &       &      &       \\
%\hline
%\hline
%\end{tabular}
%\end{table}
%
%\it{GTO (no GIAOs) calculations using Dalton}
%
%\end{frame}
