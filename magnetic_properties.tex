\begin{frame}
    \frametitle{Magnetic properties}
    \centering
    \textbf{The electronic Hamiltonian in an external magnetic field}
    \begin{align}
        \nonumber
        \hat{H}(\bs{B},\bs{M}) =& \frac{1}{2}\sum_i\Big(-i\nabla + \bs{A}\Big)^2
        -\sum_i\bs{m}_i\cdot\bs{B}(\bs{r_i}) - \sum_{iK}\frac{Z_K}{r_{iK}}
        +\frac{1}{2}\sum_{i \neq j}\frac{1}{r_{ij}}\\
        \nonumber
        & + \frac{1}{2}\sum_{K \neq L}
        \frac{Z_K Z_L}{R_{KL}} - \sum_K\bs{M}_K\cdot\bs{B}(\bs{R}_K) +
        \sum_{K>L}\bs{M}_K^TD_{KL}\bs{M}_L
    \end{align}

    \vspace{10mm}

    \textbf{The vector potential defines the magnetic field}
    \begin{equation}
        \nonumber
        \bs{B}(\bs{r}_i) = \nabla_i \times \bs{A}(\bs{r}_i)
    \end{equation}

    \vspace{10mm}

    \textbf{The vector potential depends on the choice of origin,\\
    but the magnetic field does not}
\end{frame}

\begin{frame}
    \frametitle{Magnetic properties}
    \begin{exampleblock}{{
        \begin{center}
        \it{''
        In theory, theory and practice are the same,\\
        in practice they are not
        ''}
        \end{center}
        }}
	\vskip2mm
	\hspace*\fill{--- Albert Einstein (?)}
    \end{exampleblock}

    \pause
    \vspace{10mm}

    \centering
    \textbf{The gauge origin problem}\\
    In theory not a problem, the physics is\\
    independent of the choice of origin\\

    \vspace{5mm}

    In practice the chioce of origin affects the quality of the\\
    results whenever \textbf{incomplete} basis sets are used
\end{frame}

\begin{frame}
    \frametitle{Magnetic properties}
    \centering
    \textbf{Several solutions proposed}

    \begin{columns}
    \begin{column}[b]{0.1\linewidth}
    \end{column}
    \begin{column}[b]{0.9\linewidth}
    \begin{itemize}
        \item   Individual Gauge for Localized Orbitals (IGLO)
        \item   Localized Orbitals/Local Origin (LORG)
        \item   Gauge-Including Atomic Orbitals (GIAO or London AOs)
    \end{itemize}
    \end{column}
    \end{columns}

    \vspace{5mm}

    \textbf{London Atomic Orbitals}
    \begin{equation}
        \nonumber
        \chi_\mu(\bs{r}) =
        e^{(-i/2)(\bs{B}\times[\bs{R}_\mu-\bs{R}_O]\cdot\bs{r})}\chi_\mu^{(0)}(\bs{r})
    \end{equation}

    \vspace{5mm}

    \textbf{History}
    \begin{columns}
    \begin{column}[b]{0.1\linewidth}
    \end{column}
    \begin{column}[b]{0.9\linewidth}
    \begin{itemize}
        \item   IGLO most popular in the 80's
        \item   London orbitals introduced already in the 30's,\\
                but first efficient implementation in 90's
        \item   London orbitals completely dominating last 20 years
    \end{itemize}
    \end{column}
    \end{columns}
\end{frame}
\begin{frame}
\frametitle{Magnetic properties}

\centering
\textbf{Second order property}
\begin{equation}
    \nonumber
    M = \left. \frac{\ud^2 E}{\ud b \ud a}\right|_{a,b=0}
\end{equation}

\vspace{8mm}

\begin{columns}
\begin{column}[b]{0.2\linewidth}
\end{column}

\begin{column}[b]{0.6\linewidth}
\begin{itemize}
    \item External magnetic field $\boldsymbol{B}$
    \item Nuclear magnetic moment $\boldsymbol{M}_K$ of nucleus $K$
\end{itemize}
\end{column}

\begin{column}[b]{0.2\linewidth}
\end{column}
\end{columns}

\vspace{13mm}

\begin{columns}
\begin{column}[b]{0.3\linewidth}
    \centering
    \textbf{Magnetizability tensor}
    \begin{equation}
        \nonumber
        \xi = \left. \frac{\ud^2 E}{\ud \boldsymbol{B} \ud \boldsymbol{B}}\right|_{\boldsymbol{B}=0}
    \end{equation}
\end{column}

\begin{column}[b]{0.4\linewidth}
    \centering
    \textbf{NMR shielding tensor}
    \begin{equation}
        \nonumber
        \sigma_K = \left. \frac{\ud^2 E}{\ud \boldsymbol{M}_K \ud
        \boldsymbol{B}}\right|_{\boldsymbol{M}_K,\boldsymbol{B}=0}
    \end{equation}
\end{column}

\begin{column}[b]{0.4\linewidth}
    \centering
    \textbf{Spin-spin coupling tensor}
    \begin{equation}
        \nonumber
        K_{KL} = \left. \frac{\ud^2 E}{\ud \boldsymbol{M}_K\ud
        \boldsymbol{M}_L}\right|_{\boldsymbol{M}_K,\boldsymbol{M}_L=0}
    \end{equation}
\end{column}
\end{columns}

\end{frame}


\begin{frame}
\frametitle{Magnetic properties}

\begin{columns}
\begin{column}[b]{0.48\linewidth}
    \centering
    \textbf{First-order interaction}
    \begin{equation}
        \nonumber
        \hat{h}^{(a)} = 
        \left. \frac{\ud \hat{H}}{\ud a}\right|_{a=0}
    \end{equation}
\end{column}

\begin{column}[b]{0.48\linewidth}
    \centering
    \textbf{Second-order interaction}
    \begin{equation}
        \nonumber
        \hat{h}^{(a,b)} = 
        \left. \frac{\ud^2 \hat{H}}{\ud b \ud a}\right|_{a,b=0}
    \end{equation}
\end{column}
\end{columns}

\vspace{5mm}

\centering
\textbf{First-order (paramagnetic) operators}
\begin{equation}
    \nonumber
    \hat{h}^{(B)} = \frac{1}{2}\sum_j^{el} \hat{l}_{jO} \qquad \qquad \qquad
    \hat{h}^{(M_K)} = \alpha^2 \sum_j^{el} \frac{\hat{l}_{jK}}{r_{jK}^3}
\end{equation}

\vspace{5mm}

\textbf{Second-order (diamagnetic) operators}
\begin{align}
    \nonumber
    \hat{h}^{(B,B)} &= 
    \sum_j^{el} \left(r_{jO}\cdot r_{jO}\right)1 - r_{jO}r_{jO}^T\\
    \nonumber
    \hat{h}^{(B,M_K)} &= \frac{\alpha^2}{2} 
    \sum_j^{el} \frac{\left(r_{jO}\cdot r_{jK}\right)1 -
    r_{jO}r_{jK}^T}{r_{jK}^3}
\end{align}

\end{frame}

\begin{frame}
\frametitle{Magnetic properties}
\centering
\textbf{Traditional formulation}
\begin{equation}
    \nonumber
    M = \bra{0}\pert{\hamiltonian}{a,b}\ket{0} - 2\sum_{n \neq 0}
    \frac{\bra{0}\pert{\hamiltonian}{a}\ket{n}\bra{n}\pert{\hamiltonian}{b}\ket{0}}
    {E_n - E_0}
\end{equation}

\vspace{7mm}

\textbf{Density matrix formulation}
\begin{equation}
    \nonumber
    M = Tr\Big[\pert{D}{0}\pert{h}{a,b}\Big] + Tr\Big[\pert{D}{a}\pert{h}{b}\Big]
\end{equation}

\vspace{7mm}

\textbf{Real-space formulation}
\begin{equation}
    \nonumber
    M = 
    \int \pert{\density}{0}\pert{\hamiltonian}{a,b} \ud r +
    \int \pert{\density}{a}\pert{\hamiltonian}{b} \ud r
\end{equation}
\begin{equation}
    \nonumber
    M = \sum_i
    \bra{\pert{\orbital_i}{0}}\pert{\hamiltonian}{a,b}\ket{\pert{\orbital_i}{0}} 
    + \sum_i \bigg[
    \bra{\pert{\orbital_i}{0}}\pert{\hamiltonian}{b}\ket{\pert{\orbital_i}{a}} +
    \bra{\pert{\orbital_i}{a}}\pert{\hamiltonian}{b}\ket{\pert{\orbital_i}{0}}
    \bigg]
\end{equation}

\end{frame}

\begin{frame}
\frametitle{Magnetic properties}
\centering
\textbf{Magnetizability tensor}
\begin{align}
    \nonumber
    \xi_{\mu\nu}
    &= \int \pert{\density}{0}\pert{\hamiltonian}{B,B}_{\mu\nu} \ud r
    +  \int \pert{\density}{B}_{\mu}\pert{\hamiltonian}{B}_{\nu} \ud r
\end{align}

\vspace{5mm}

\textbf{NMR shielding tensor}
\begin{align}
    \nonumber
    \big[\sigma_K\big]_{\mu\nu}
    &= \int \pert{\density}{0}\pert{\hamiltonian}{B,M_K}_{\mu\nu} \ud r
    +  \int \pert{\density}{B}_{\mu}\pert{\hamiltonian}{M_K}_{\nu} \ud r
\end{align}

\vspace{5mm}

\begin{itemize}
    \item   $\mu,\nu=x,y,z$ are components of the perturbing fields
    \item   $\xi$ and $\sigma_K$ of \textbf{all} nuclei computed from the
            \textbf{same} perturbed density $\pert{\density}{B}$
    \item   $\pert{\density}{B}_\mu$ means density perturbed by
            $\pert{\hamiltonian}{B}_\mu$, e.i. the magnetic field component $B_\mu$
    \item   For NMR shieldings, the order of the perturbations can be swapped
\end{itemize}
\end{frame}
