
\section{Front page}
\section{Outlook}
\section{Multiwavelets}
\section{Decreasing order}
\section{Parallelization}
\subsection{Introduction}
So why do we need parallel processing? There are basically two reasons for that: either to
get more processing power, or to get more available memory (or perhaps both at once). 
Computational scientists, and quantum chemists in particular, are always trying to push the 
limits of how accurately a particular property can be computed or how big a system that can 
be treated with a given computational method, and this means pushing the computational resources.

Every computer available today will have at least a dual-core processor, and the machines at
modern high performance computing centres will typically have 8, 12, 16 or 20 processors
on each machine, so-called host, but even if these processors share the same memory, they
will not automatically work together on the same problem, as at most one processor can 
work on a given application at a time. While this is fine for the day to day work at the 
office, as you usually have many different applications running at the same time, it is much 
more problematic for computational scientists, as we need to get as much computational power 
as possible for a single application program. This means that we have to make several 
independent computers work in parallel in solving the same problem, and this is challenging 
in many respects. 

So what does parallelization mean? It means distributing work between the available processors.
This needs to done in a balanced manner, so that each processor gets an approximately equal 
amount of work to execute. It means syncronizing the different processors so that the arithmetic
operations are performed in the correct order. If some of the processors does NOT share the same
memory, the data needs to be distributed among the different hosts, and at some point we usually
need to communicate some data between the hosts.


Heisenbugs. If you're a quantum chemist you get the joke behind that name, but these are 
particularly nasty bugs or errors in your code that tends to change character or even dissapear
when you go looking for them.

\section{Coulomb interaction}
\section{Chemistry}
\section{Real-space DFT}
\section{Acknowledgments}

