%\begin{frame}
%    \centering
%    \textbf{\Large{Multiwavelets}}
%\end{frame}

%\begin{frame}
%    \frametitle{Multiwavelets}
%    \scriptsize
%    \begin{columns}
%    \begin{column}[b]{0.65\linewidth}
%    \begin{itemize}
%        \item   \textbf{Scaling functions} are polynomials or order $\leq k$
%        \item   \textbf{Dilation and translation} to refinement scale $n$
%	        \begin{equation}
%		    \nonumber
%		    \phi_l^n(x) = 2^{n/2}\phi(2^nx-l)
%	        \end{equation}
%        \item   At scale $n$ there are $2^n$ subintervals
%        \item   The length of each subinterval is $2^{-n}$
%        \item   \textbf{Scaling projection} at scale $N$
%	        \begin{equation}
%		    \nonumber
%		    f(x) \approx f^N(x) = \sum_l s_l^N \phi_l^N(x)
%	        \end{equation}
%        \item   For a $k$-order basis in $d$ dimensions there are\\
%	        $\left(2(k+1)\right)^{nd}$ basis functions at scale $n$\\
%    \end{itemize}
%    \centering
%    \includegraphics[scale=0.2]{figures/unifgrid.pdf}
%    \end{column}
%    \begin{column}[b]{0.35\linewidth}
%    \centering
%    \includegraphics[scale=0.4, clip, viewport=150 450 450 750]
%        {figures/scaling.pdf}\\
%    \includegraphics[scale=0.3, clip, viewport=100 400 500 800]
%        {figures/refinement.pdf}
%    \end{column}
%    \end{columns}
%\end{frame}

\begin{frame}
    \frametitle{Multiwavelets}
    \scriptsize
    \begin{columns}
    \begin{column}[b]{0.65\linewidth}
    \begin{itemize}
        \item   \textbf{Scaling functions} are polynomials or order $\leq k$
        \pause
        \item   \textbf{Dilation and translation} to refinement scale $n$
	        \begin{equation}
		    \nonumber
		    \phi_l^n(x) = 2^{n/2}\phi(2^nx-l)
	        \end{equation}
        \item   At scale $n$ there are $2^n$ subintervals
        \item   The length of each subinterval is $2^{-n}$
        \item   \textbf{Scaling projection} at scale $N$
	        \begin{equation}
		    \nonumber
		    f(x) \approx f^N(x) = \sum_l s_l^N \phi_l^N(x)
	        \end{equation}
        \pause
        \item   For a $k$-order basis in $d$ dimensions there are\\
	        $\left(2(k+1)\right)^{nd}$ basis functions at scale $n$\\
    \end{itemize}
    \end{column}
    \begin{column}[b]{0.35\linewidth}
    \centering
    \only<1>{
        \includegraphics[scale=0.4, clip, viewport=150 450 450 750] {figures/scaling.pdf}
        \vspace{3mm}
    }
    \only<2>{
        \includegraphics[scale=0.30, clip, viewport=100 400 500 800] {figures/refinement.pdf}
        \vspace{3mm}
    }
    \only<3>{
        \includegraphics[scale=0.25]{figures/unifgrid.pdf}
        \vspace{3mm}
    }
    \end{column}
    \end{columns}
\end{frame}

%\begin{frame}
%    \frametitle{Multiwavelets}
%    \centering
%    \includegraphics[scale=0.5, clip, viewport=150 450 450 750]{figures/scaling.pdf}
%\end{frame}

%\begin{frame}
%    \frametitle{Multiwavelets}
%    \centering
%    \includegraphics[scale=0.5, clip, viewport=100 400 500 800]{figures/refinement.pdf}
%\end{frame}

%\begin{frame}
%    \frametitle{Multiwavelets}
%    \centering
%    \includegraphics[scale=0.35]{figures/unifgrid.pdf}
%\end{frame}

\begin{frame}
    \frametitle{Multiwavelets}
    \centering
    \only<1>{\ \ \ \includegraphics[clip, viewport=50 100 600 800, scale=0.5]{figures/f0.pdf}}
    \only<2>{\ \ \includegraphics[clip, viewport=50 100 600 800, scale=0.5]{figures/f1.pdf}}
    \only<3>{\ \includegraphics[clip, viewport=50 100 600 800, scale=0.5]{figures/f2.pdf}}
    \only<4>{\includegraphics[clip, viewport=50 100 600 800, scale=0.5]{figures/f3.pdf}}
    \only<5>{\includegraphics[clip, viewport=50 100 600 800, scale=0.5]{figures/f4.pdf}}
\end{frame}

\begin{frame}
    \frametitle{Multiwavelets}
    \scriptsize
    \begin{columns}
    \begin{column}[b]{0.58\linewidth}
    \begin{itemize}
        \item   \textbf{Wavelet projection} at scale $N$
	    \begin{equation}
	        \nonumber
	        df^n(x) = f^{n+1}(x) - f^{n}(x)
	    \end{equation}
        \item   \textbf{Multiresolution} representation
	    \begin{equation}
	        \nonumber
	        f^N(x) = f^{0}(x) + \sum_{n=0}^{N-1} df^{n}(x)
	    \end{equation}
        \pause
        \item   \textbf{Adaptive refinement} by local thresholding
	    \begin{equation}
	        \nonumber
	        \|df_l^n\| < \frac{\epsilon}{2^{n/2}}\|f\|
	    \end{equation}
        \item   \textbf{Guaranteed precision} $\epsilon$
    \end{itemize}
    \vspace{3mm}
    \end{column}
    \begin{column}[b]{0.42\linewidth}
    \centering
    \only<1>{
        \includegraphics[scale=0.35, clip, viewport=100 400 500 800] {figures/refinement.pdf}
        \vspace{2mm}
    }
    \only<2>{
        \includegraphics[scale=0.35, clip, viewport=100 400 500 800] {figures/adaptivity.pdf}
        \vspace{2mm}
    }
    \only<3>{
        \includegraphics[scale=0.15]{figures/adapgrid.pdf}
        \vspace{10mm}
    }
    \end{column}
    \end{columns}
\end{frame}

%\begin{frame}
    %\frametitle{Multiwavelets}
    %\centering
    %\includegraphics[scale=0.5, clip, viewport=100 400 500 800]{figures/adaptivity.pdf}
%\end{frame}

%\begin{frame}
    %\frametitle{Multiwavelets}
    %\centering
    %\includegraphics[scale=0.3]{figures/unifgrid.pdf}
    %\includegraphics[scale=0.1788]{figures/adapgrid.pdf}
%\end{frame}

